%%!!ATTENZIONE!!
%QUESTO FILE E' UN ESEMPIO. COPIATE ALL'INTERNO DELLA CARTELLA DEL DOCUMENTO SU CUI STATE LAVORANDO E MODIFICATE
%% NON MODIFICARE QUESTE IMPOSTAZIONI
\date{}
\usepackage[latin9]{inputenc}
\usepackage{fancyhdr}
\usepackage{ucs}
\usepackage{lastpage}
\usepackage{lscape}
\usepackage{longtable}
\usepackage[colorlinks=true]{hyperref}

%% DA QUI PER I LISTATI PIU' LUNGHI DI UNA PAGINA

\usepackage{afterpage}
\usepackage{listings}
\makeatletter
\newbox\loc@box
\newenvironment{longonecolumn}{%
        \ifvoid\loc@box\else
                \errmessage{Too many longonecolumn environments too
                close together.}%
        \fi
        \global\setbox\loc@box\vbox\bgroup
                \hsize\textwidth
                \columnwidth\hsize
                \linewidth\hsize
                \@twocolumnfalse
}{%
        \egroup
        \afterpage\loc@placebox
}
\newcommand*\loc@placebox{%
        \if@firstcolumn
                \loc@placebox@
        \else
                \afterpage\loc@placebox
        \fi
}
\newcommand*\loc@placebox@{%
        \onecolumn
        \begingroup
                \loop\ifdim\ht\loc@box>\vsize
                        \setbox\z@\vsplit\loc@box to\vsize
                        \unvbox\z@
                        \vfill
                        \pagebreak
                \repeat
                \ifdim\ht\loc@box>\dbltopfraction\vsize
                        \unvbox\loc@box
                        \vfill
                        \twocolumn\relax
                \else
                       \twocolumn[\unvbox\loc@box\vspace\dbltextfloatsep]%
                \fi
        \endgroup
        \global\setbox\loc@box\box\voidb@x % Probably not needed
        \onecolumn
}
\makeatother

%% RIPRENDE NORMALMENTE

\hypersetup{
    colorlinks,
    citecolor=black,
    filecolor=black,
    linkcolor=black,
    urlcolor=black
}
\pagestyle{fancy}
\fancyhead{}
\fancyfoot{}
\newcommand{\HRule}{\rule{\linewidth}{0.5mm}}
\fancyhead[RE, RO]{\doctitle \versiondoc - \NomeEnte}
%%DOVETE COPIARE IL FILE LogoTeam.jpg ALL'INTERNO DELLA CARTELLA IMG (CHE DOVRETE CREARE) DENTRO LA CARTELLA DEL VOSTRO DOCUMENTO
\lhead{\setlength{\unitlength}{1mm}
	\begin{picture}(0,0)
		\put(5,0){\includegraphics[scale=0.07]{img/isees}}
	\end{picture}}
\fancyfoot[CE, CO]{\thepage\ di \pageref{LastPage}}

%%Comandi particolari per i nomi delle figure. Chiamare come una funzione LaTeX
\newcommand{\NomeEnte}{\emph{Ass. ISEES }}

%%DA QUI IN POI POTETE MODIFICARE

%% INSERIRE QUI IL NOME DEL DOCUMENTO (INSERITE SEMPRE UNO SPAZIO ALLA FINE DEL NOME)
\newcommand{\doctitle}{Manuale installazione LiquidFeedback }

%% INSERIRE QUI LA VERSIONE ATTUALE DEL DOCUMENTO (INSERITE SEMPRE UNO SPAZIO ALLA FINE DELLA VERSIONE)
\newcommand{\versiondoc}{V0.13 }

%%INSERITE QUI LA DATA DI COMPILAZIONE FINALE DEL DOCUMENTO
\newcommand{\datared}{GG/MM/AAAA}

%%INSERIRE QUI IL/I REDATTORI
\newcommand{\redattore}{\begin{itemize}
\item Giorgio Maggiolo
\item Jacopo Amistani Guarda
\end{itemize}}

%%INSERIRE IL/I NOME DEI VERIFICATORI CHE HANNO VERIFICATO IL DOCUMENTO
\newcommand{\verificatori}{xxxxxx}

%%INSERIRE IL NOME DI CHI HA APPROVATO IL DOCUMENTO
\newcommand{\approvazione}{xxxxxx}

%%INSERIRE LA TIPOLOGIA DI USO DEL DOCUMENTO [Interno/Esterno]
\newcommand{\usodoc}{Esterno}

%%INSERIRE LA LISTA DI DISTRIBUZIONE DEL DOCUMENTO
\newcommand{\listadistr}{ }

%%INSERIRE IL SOMMARIO DEL DOCUMENTO
\newcommand{\testosommario}{Il seguente documento ha lo scopo di illustrare quali operazioni dovr\'a eseguire un installatore per installare correttamente il sistema Liquid Feedback all'interno del proprio server}

